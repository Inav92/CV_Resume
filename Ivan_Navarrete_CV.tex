\documentclass[latterpaper]{article}
\usepackage[utf8]{inputenc}

%\title{CV_01}
%\author{ivan.navarrete1992 }
%\date{August 2020}

\usepackage{polyglossia}
%%%si usamos compiladores más modernos como xelatex o lualatex, podemos usar el paquete polyglossia1. Este paquete es más simple y ligero aprovechando que estos compiladores ya gestionan la codificación y las fuentes de por sí. En este caso no cargamos el idioma como opción, sino que lo elegimos con el comando \setmainlanguage:
\setmainlanguage{spanish}

\usepackage{parskip} %para generar espacios en blanco, no necesario agregar \\

\usepackage{color, graphicx} %¿colores? aqui la duda es si es necesario usar \RequirePackage{color,graphicx}

%Setup hyperref package, and colours for links
\usepackage{hyperref}
\definecolor{linkcolour}{rgb}{0,0.2,0.6}
\hypersetup{colorlinks,breaklinks,urlcolor=linkcolour, linkcolor=linkcolour}

%otra manera que quizas sea buena:
%\usepackage{hyperref}
%\hypersetup{
%%    colorlinks=true,
%%    linkcolor=cyan,
%%    filecolor=magenta,
%%    urlcolor=blue,
%%}

\usepackage{fullpage} %Ajuste a la pagina, la idea es buscar un ajuste correcto al tipo carta
\usepackage{titlesec}	
\titleformat{\section}{\Large\scshape\raggedright}{}{0em}{}[\titlerule]
\titlespacing{\section}{0pt}{3pt}{3pt}

%\usepackage{amssymb}
\usepackage{fontawesome} %Esto permite usar los iconos

\usepackage{multicol}
\newenvironment{twocollist}{
    \begin{multicols}{2}
    \begin{itemize}
    }{
    \end{itemize}
    \end{multicols}
}

\begin{document}


\pagestyle{empty} %ante modificaciones para formato de pagina

\par{\centering
		{\Huge \textsc{Iván Navarrete Palomera} %\textsc{} resulta en palabras con mayusculas"
	}\par}	
	\rule{\textwidth}{0.1pt}


    \begin{center}
        Ingeniero Civil en Obras civiles - Universidad de Santiago de Chile \newline\newline
        \begin{tabular}{ll}
            \faGithub~\href{https://github.com/Inav92}{github.com/Inav92} & \faAt~\href{mailto:ivan.navarrete1992@gmail.com}{\texttt{ivan.navarrete1992@gmail.com}} \\
            \faMapMarker~    Santiago RM, Chile & \faPhone~+56 949 772 860 \\
        \end{tabular}
        \newline \newline
        Descargar la ultima versión de este documento 
        \href{https://github.com/Inav92}{\faDownload}.
        (Ultima actualización: \today)
%        \rule{\textwidth}{0.1pt}
%
    \end{center}
	
\section{Resumen y Objetivos}


\par{Profesional altamente motivado, titulado de Ingeniería Civil en Obras Civiles en Octubre del 2018, en búsqueda de continuar su aprendizaje y experiencia en empresas que le permitan crecer profesionalmente, donde pueda encontrar una guía y contribuir con su pasión a la Ingeniería. Con gran deseo de incorporarse a equipos de trabajo multidisciplinarios y diversificados, de trabajar en proyectos de Construcción, Mineros o Viales, que se diferencien, dentro o fuera de la Región Metropolitana.
\textbf{Disponibilidad inmediata}.}
	
% \rule{\textwidth}{0.1pt}
		
	
\section{Experiencia Laboral}
	
	
%\begin{center}
    

    \begin{tabular}{r | p{13.0 cm}}
%\emph{PALABRA}: enfasis

        \emph{05/2019 - 03/2020} & \textsc{\textbf{CEMOSA}} \emph{\textbf{Agencia en Chile - Laboratorio en Obra}}\\\emph{(10 meses)}&
         %\footnotesize{ 
        \par{\textbf{Ingeniero en laboratorio de materiales}, faena hidroeléctrica ``Los Cóndores'' Ruta CH-115, San Clemente, Región del Maule. Obra a cargo de la constructora FERROVIAL AGROMAN, que compromete 16 km de túneles excavados con TBM, minería convencional, edificación  y trabajos afines en circunstancias climáticas adversas.}

        %\newline{}
\textbf{$\nabla$ Trabajos desarrollados - Conocimientos adquiridos:}
  \begin{itemize}
\item Normativa nacional e internacional aplicada a obras civiles, viales y de minería
\item Ensayos de laboratorio y controles  \emph{in-situ}
\item Creación de planillas de cálculo, programas, informes y protocolos de gestión 
\item Dosificación y cuidados de hormigones de alta resistencia (>70 MPa)
\item Métodos constructivos de alta tecnología 
\item Administración de personal y formación de grupos de trabajo
 \end{itemize}  
% }



 \\\multicolumn{2}{c}{} \\

\emph{Enero - Marzo} & \textsc{\textbf{CEMOSA}} \emph{\textbf{Agencia en Chile}}\\\emph{2018}&
 %\footnotesize{ 
Faena hidroeléctrica ``Los Cóndores'' Ruta CH-115, San Clemente, Región del Maule.
Allí colabora con la ejecución de informes y ensayos de laboratorio de materiales.
%}
% }
 \\\multicolumn{2}{c}{} \\

	
\emph{Enero - Marzo} &\textsc{\textbf{CEMOSA}} \emph{\textbf{Agencia en Chile}} \\\emph{2017}& 
%\footnotesize{
Realiza \textbf{práctica profesional de Ingeniería Civil} en CEMOSA, \emph{Centro de Estudios de Materiales de Obra}, laboratorio de materiales.
%}
\\\multicolumn{2}{c}{} \\


\emph{Enero - Junio} & \textbf{\textsc{BUDNIK S.A.}} \emph{Baldosas, prefabricados de hormigón y deco-hogar}\\\emph{2011}&
%\footnotesize{
Lleva a cabo su \textbf{práctica de Contador técnico de nivel medio}, experiencia que le otorga conocimientos en la gestión de compras y ventas, documentos contables y procedimientos tributarios.
%}	
	
\end{tabular}	


%\rule{\textwidth}{0.1pt}


\section{Educación}
	
\begin{tabular}{r p{14 cm}}	
\emph{2012 - 2018} & \textbf{\textsc{Universidad de Santiago de Chile}} | \emph{Titulado de Ingeniería Civil en Obras Civiles} \\

&Completa sus estudios de pregrado, graduándose como Ingeniero Civil en Obras Civiles. Obteniendo una calificación máxima en su examen de grado.\\
&\textbf{Tesis}: \emph{``Modelación computacional del comportamiento histerético de conexiones en paneles de madera contra-laminada''} | Profesor guía: PhD. Erick Saavedra Flores \textsc{\textit{PhD in Civil Engineering (Swansea, Wales, UK).}}\\\multicolumn{2}{c}{}
\end{tabular}
	
\begin{tabular}{r p{14,0 cm}}	
\emph{2007 - 2010} & \textbf{\textsc{Instituto Superior de Comercio Eduardo Frei Montalva}}\\
& Finaliza sus estudios de enseñanza media, graduándose como contador técnico de nivel medio.

\end{tabular}

\begin{tabular}{r p{14,0 cm}}	
\emph{Académicas} & \textbf{\textsc{Ayudante, curso de Diseño en Hormigón Armado}}\\  
&~Se desempeña como ayudante del curso de Hormigón Armado I, impartido en la Universidad de Santiago, para alumnos de Ingeniería Civil en Obras Civiles (plan vespertino). En él logra consolidar y brindar sus conocimientos sobre el diseño con hormigón armado aplicando las normativas naciones e internacionales (A.C.I. 318).\\\\
\end{tabular}


\begin{tabular}{r p{14.0 cm}}	
\emph{\phantom{cincó}Cursos} & \textbf{\textsc{Seismology Skill Building Workshop (06/2020 - 09/2020)}}\\
& Participa de curso de Sismología, impartido por la institución IRIS (info-web: \url{https://www.iris.edu/hq/short-courses/course/ssb_workshop}) con fin de ampliar sus conocimientos respecto a los terremotos. El programa incluye los siguientes tópicos:\\
&
\begin{itemize}
\item Introducción a la computación científica y estrategias de programación
\item Grabaciones sismológicas y análisis de códigos sismográficos
\item Acceso, herramientas de visualización y manipulación de datos sísmicos
\item Eliminación de ruido, correlación de datos e inducción de sismicidad
  \end{itemize}\\
  \end{tabular}

\section{Aptitudes y Competencias}

%\begin{tabular}{r p{13.3 cm}}
  \begin{twocollist}
            \item \faLanguage~Inglés B1+ (Intermedio) actualmente cursando estudios, en vías de mejorar su nivel de Inglés \href{https://github.com/Inav92}{(link al certificado)}, preparando examen IELTS.
            \item \faFileExcelO~\texttt{Excel} nivel avanzado.
            \item \faCode~\texttt{SAP2000}, \texttt{Python} \href{https://coursera.org/share/07ea9a609345e33b5b1b5b0e33925258}{(ver certificado)}, \texttt{ANSYS}, \texttt{ETABS}. Para la modelación computacional.
            \item \faLinux~GNU/Linux usuario entusiasta.
            \item Siempre buscando aprender cosas nuevas.
            \item Con buenas practicas comunicativas, logrando esclarecer sus ideas.
            \item Respetuoso y colaborativo, con adaptabilidad a diversos ambientes laborales.
            \item Buena capacidad de análisis y resolución de problemas.
  \end{twocollist}



\section{Intereses y Actividades}

  \begin{twocollist}
            \item Modelación estructural, Matemática, Física y ciencia de los materiales.
            \item El empleo de nuevos materiales y energías no convencionales.
            \item Obras Viales y Edificación.
            \item Diseño en acero y hormigón armado.
            \item Minería, mecánica de suelos y rocas.
            \item Programación y manejo de datos.
            \item Interprete aficionado de piano clásico.
            \item Deportes: trekking y natación.
            \item Licencia de conductor clase B desde 2015

  \end{twocollist}






\end{document}
